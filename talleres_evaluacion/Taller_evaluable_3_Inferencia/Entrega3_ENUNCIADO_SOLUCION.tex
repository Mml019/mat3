% Options for packages loaded elsewhere
\PassOptionsToPackage{unicode}{hyperref}
\PassOptionsToPackage{hyphens}{url}
\PassOptionsToPackage{dvipsnames,svgnames*,x11names*}{xcolor}
%
\documentclass[
]{article}
\usepackage{lmodern}
\usepackage{amssymb,amsmath}
\usepackage{ifxetex,ifluatex}
\ifnum 0\ifxetex 1\fi\ifluatex 1\fi=0 % if pdftex
  \usepackage[T1]{fontenc}
  \usepackage[utf8]{inputenc}
  \usepackage{textcomp} % provide euro and other symbols
\else % if luatex or xetex
  \usepackage{unicode-math}
  \defaultfontfeatures{Scale=MatchLowercase}
  \defaultfontfeatures[\rmfamily]{Ligatures=TeX,Scale=1}
\fi
% Use upquote if available, for straight quotes in verbatim environments
\IfFileExists{upquote.sty}{\usepackage{upquote}}{}
\IfFileExists{microtype.sty}{% use microtype if available
  \usepackage[]{microtype}
  \UseMicrotypeSet[protrusion]{basicmath} % disable protrusion for tt fonts
}{}
\makeatletter
\@ifundefined{KOMAClassName}{% if non-KOMA class
  \IfFileExists{parskip.sty}{%
    \usepackage{parskip}
  }{% else
    \setlength{\parindent}{0pt}
    \setlength{\parskip}{6pt plus 2pt minus 1pt}}
}{% if KOMA class
  \KOMAoptions{parskip=half}}
\makeatother
\usepackage{xcolor}
\IfFileExists{xurl.sty}{\usepackage{xurl}}{} % add URL line breaks if available
\IfFileExists{bookmark.sty}{\usepackage{bookmark}}{\usepackage{hyperref}}
\hypersetup{
  pdftitle={Entrega 3 Problemas y Talleres MATIII Estadística grado informática 2019-2020},
  pdfauthor={Ricardo Alberich},
  colorlinks=true,
  linkcolor=red,
  filecolor=Maroon,
  citecolor=blue,
  urlcolor=blue,
  pdfcreator={LaTeX via pandoc}}
\urlstyle{same} % disable monospaced font for URLs
\usepackage[margin=1in]{geometry}
\usepackage{color}
\usepackage{fancyvrb}
\newcommand{\VerbBar}{|}
\newcommand{\VERB}{\Verb[commandchars=\\\{\}]}
\DefineVerbatimEnvironment{Highlighting}{Verbatim}{commandchars=\\\{\}}
% Add ',fontsize=\small' for more characters per line
\usepackage{framed}
\definecolor{shadecolor}{RGB}{248,248,248}
\newenvironment{Shaded}{\begin{snugshade}}{\end{snugshade}}
\newcommand{\AlertTok}[1]{\textcolor[rgb]{0.94,0.16,0.16}{#1}}
\newcommand{\AnnotationTok}[1]{\textcolor[rgb]{0.56,0.35,0.01}{\textbf{\textit{#1}}}}
\newcommand{\AttributeTok}[1]{\textcolor[rgb]{0.77,0.63,0.00}{#1}}
\newcommand{\BaseNTok}[1]{\textcolor[rgb]{0.00,0.00,0.81}{#1}}
\newcommand{\BuiltInTok}[1]{#1}
\newcommand{\CharTok}[1]{\textcolor[rgb]{0.31,0.60,0.02}{#1}}
\newcommand{\CommentTok}[1]{\textcolor[rgb]{0.56,0.35,0.01}{\textit{#1}}}
\newcommand{\CommentVarTok}[1]{\textcolor[rgb]{0.56,0.35,0.01}{\textbf{\textit{#1}}}}
\newcommand{\ConstantTok}[1]{\textcolor[rgb]{0.00,0.00,0.00}{#1}}
\newcommand{\ControlFlowTok}[1]{\textcolor[rgb]{0.13,0.29,0.53}{\textbf{#1}}}
\newcommand{\DataTypeTok}[1]{\textcolor[rgb]{0.13,0.29,0.53}{#1}}
\newcommand{\DecValTok}[1]{\textcolor[rgb]{0.00,0.00,0.81}{#1}}
\newcommand{\DocumentationTok}[1]{\textcolor[rgb]{0.56,0.35,0.01}{\textbf{\textit{#1}}}}
\newcommand{\ErrorTok}[1]{\textcolor[rgb]{0.64,0.00,0.00}{\textbf{#1}}}
\newcommand{\ExtensionTok}[1]{#1}
\newcommand{\FloatTok}[1]{\textcolor[rgb]{0.00,0.00,0.81}{#1}}
\newcommand{\FunctionTok}[1]{\textcolor[rgb]{0.00,0.00,0.00}{#1}}
\newcommand{\ImportTok}[1]{#1}
\newcommand{\InformationTok}[1]{\textcolor[rgb]{0.56,0.35,0.01}{\textbf{\textit{#1}}}}
\newcommand{\KeywordTok}[1]{\textcolor[rgb]{0.13,0.29,0.53}{\textbf{#1}}}
\newcommand{\NormalTok}[1]{#1}
\newcommand{\OperatorTok}[1]{\textcolor[rgb]{0.81,0.36,0.00}{\textbf{#1}}}
\newcommand{\OtherTok}[1]{\textcolor[rgb]{0.56,0.35,0.01}{#1}}
\newcommand{\PreprocessorTok}[1]{\textcolor[rgb]{0.56,0.35,0.01}{\textit{#1}}}
\newcommand{\RegionMarkerTok}[1]{#1}
\newcommand{\SpecialCharTok}[1]{\textcolor[rgb]{0.00,0.00,0.00}{#1}}
\newcommand{\SpecialStringTok}[1]{\textcolor[rgb]{0.31,0.60,0.02}{#1}}
\newcommand{\StringTok}[1]{\textcolor[rgb]{0.31,0.60,0.02}{#1}}
\newcommand{\VariableTok}[1]{\textcolor[rgb]{0.00,0.00,0.00}{#1}}
\newcommand{\VerbatimStringTok}[1]{\textcolor[rgb]{0.31,0.60,0.02}{#1}}
\newcommand{\WarningTok}[1]{\textcolor[rgb]{0.56,0.35,0.01}{\textbf{\textit{#1}}}}
\usepackage{longtable,booktabs}
% Correct order of tables after \paragraph or \subparagraph
\usepackage{etoolbox}
\makeatletter
\patchcmd\longtable{\par}{\if@noskipsec\mbox{}\fi\par}{}{}
\makeatother
% Allow footnotes in longtable head/foot
\IfFileExists{footnotehyper.sty}{\usepackage{footnotehyper}}{\usepackage{footnote}}
\makesavenoteenv{longtable}
\usepackage{graphicx,grffile}
\makeatletter
\def\maxwidth{\ifdim\Gin@nat@width>\linewidth\linewidth\else\Gin@nat@width\fi}
\def\maxheight{\ifdim\Gin@nat@height>\textheight\textheight\else\Gin@nat@height\fi}
\makeatother
% Scale images if necessary, so that they will not overflow the page
% margins by default, and it is still possible to overwrite the defaults
% using explicit options in \includegraphics[width, height, ...]{}
\setkeys{Gin}{width=\maxwidth,height=\maxheight,keepaspectratio}
% Set default figure placement to htbp
\makeatletter
\def\fps@figure{htbp}
\makeatother
\setlength{\emergencystretch}{3em} % prevent overfull lines
\providecommand{\tightlist}{%
  \setlength{\itemsep}{0pt}\setlength{\parskip}{0pt}}
\setcounter{secnumdepth}{5}
\renewcommand{\contentsname}{Contenidos}

\title{Entrega 3 Problemas y Talleres MATIII Estadística grado informática
2019-2020}
\author{Ricardo Alberich}
\date{13-05-2020}

\begin{document}
\maketitle

{
\hypersetup{linkcolor=blue}
\setcounter{tocdepth}{2}
\tableofcontents
}
\hypertarget{entregas-3-problemas-estaduxedstica-inferencial-1}{%
\section{Entregas 3 Problemas: Estadística Inferencial
1}\label{entregas-3-problemas-estaduxedstica-inferencial-1}}

Contestad cada GRUPO de 3 a los siguientes problemas y cuestiones en un
fichero Rmd y su salida en html o pdf.

Cambien podéis incluir capturas de problemas hechos en papel. Cada
pregunta vale lo mismo y se reparte la nota entre sus apartados.

\hypertarget{problema-1-contraste-de-paruxe1metros-de-dos-muestras.}{%
\subsection{Problema 1: Contraste de parámetros de dos
muestras.}\label{problema-1-contraste-de-paruxe1metros-de-dos-muestras.}}

Queremos comparar los tiempos de realización de un test entre
estudiantes de dos grados G1 y G2, y determinar si es verdad que los
estudiantes de G1 emplean menos tiempo que los de G2. No conocemos
\(\sigma_1\) y \(\sigma_2\). Disponemos de dos muestras independientes
de cuestionarios realizados por estudiantes de cada grado,
\(n_1=n_2=50\).

Los datos están en
\url{http://bioinfo.uib.es/~recerca/MATIIIGMAT/NotasTestGrado/}, en dos
ficheros \texttt{grado1.txt} y \texttt{grado2.txt}.

\begin{Shaded}
\begin{Highlighting}[]
\NormalTok{G1=}\KeywordTok{read.table}\NormalTok{(}\StringTok{"http://bioinfo.uib.es/~recerca/MATIIIGMAT/NotasTestGrado/grado1.txt"}\NormalTok{,}
              \DataTypeTok{header=}\OtherTok{TRUE}\NormalTok{)}\OperatorTok{$}\NormalTok{x}
\NormalTok{G2=}\KeywordTok{read.table}\NormalTok{(}\StringTok{"http://bioinfo.uib.es/~recerca/MATIIIGMAT/NotasTestGrado/grado2.txt"}\NormalTok{,}
              \DataTypeTok{header=}\OtherTok{TRUE}\NormalTok{)}\OperatorTok{$}\NormalTok{x}
\NormalTok{n1=}\KeywordTok{length}\NormalTok{(}\KeywordTok{na.omit}\NormalTok{(G1))}
\NormalTok{n2=}\KeywordTok{length}\NormalTok{(}\KeywordTok{na.omit}\NormalTok{(G2))}
\NormalTok{media.muestra1=}\KeywordTok{mean}\NormalTok{(G1,}\DataTypeTok{na.rm=}\OtherTok{TRUE}\NormalTok{)}
\NormalTok{media.muestra2=}\KeywordTok{mean}\NormalTok{(G2,}\DataTypeTok{na.rm=}\OtherTok{TRUE}\NormalTok{)}
\NormalTok{desv.tip.muestra1=}\KeywordTok{sd}\NormalTok{(G1,}\DataTypeTok{na.rm=}\OtherTok{TRUE}\NormalTok{)}
\NormalTok{desv.tip.muestra2=}\KeywordTok{sd}\NormalTok{(G2,}\DataTypeTok{na.rm=}\OtherTok{TRUE}\NormalTok{)}
\end{Highlighting}
\end{Shaded}

Calculamos las medias y las desviaciones típicas muestrales de los
tiempos empleados para cada muestra. Los datos obtenidos se resumen en
la siguiente tabla:

\[
\begin{array}{llll}
n_1&=50, & n_2&=50\\
\overline{x}_1&=9.7592926, & \overline{x}_2&=11.4660825\\
\tilde{s}_1&=1.1501225, & \tilde{s}_1&=1.5642932
\end{array}
\] Se pide:

\begin{enumerate}
\def\labelenumi{\arabic{enumi}.}
\tightlist
\item
  Comentad brevemente el código de R explicando que hace cada
  instrucción.
\item
  Contrastad si hay evidencia de que las notas medias son distintas
  entre los dos grupos. En dos casos considerando las varianzas
  desconocidas pero iguales o desconocidas pero distintas. Tenéis que
  hacer el contraste de forma manual y con funciones de \texttt{R} y
  resolver el contrate con el \(p\)-valor.
\item
  Calculad e interpretar los intervalos de confianza para la diferencia
  de medias asociados a los dos test anteriores.
\item
  Comprobad con el test de Fisher y el de Levene si las varianza de las
  dos muestras son iguales contra que son distintas. Tenéis que resolver
  el test de Fisher con \texttt{R} y de forma manual y el test de Levene
  con \texttt{R} y decidir utilizando el \(p\)-valor.
\end{enumerate}

\hypertarget{presoluciuxf3n}{%
\subsubsection{Presolución}\label{presoluciuxf3n}}

\begin{Shaded}
\begin{Highlighting}[]
\KeywordTok{var.test}\NormalTok{(G1,G2)}
\end{Highlighting}
\end{Shaded}

\begin{verbatim}
## 
##  F test to compare two variances
## 
## data:  G1 and G2
## F = 0.54057, num df = 49, denom df = 49, p-value = 0.03354
## alternative hypothesis: true ratio of variances is not equal to 1
## 95 percent confidence interval:
##  0.3067606 0.9525862
## sample estimates:
## ratio of variances 
##            0.54057
\end{verbatim}

\begin{Shaded}
\begin{Highlighting}[]
\KeywordTok{library}\NormalTok{(car)}
\end{Highlighting}
\end{Shaded}

\begin{verbatim}
## Warning: package 'car' was built under R version 3.6.3
\end{verbatim}

\begin{verbatim}
## Loading required package: carData
\end{verbatim}

\begin{verbatim}
## 
## Attaching package: 'car'
\end{verbatim}

\begin{verbatim}
## The following object is masked from 'package:dplyr':
## 
##     recode
\end{verbatim}

\begin{verbatim}
## The following object is masked from 'package:purrr':
## 
##     some
\end{verbatim}

\begin{Shaded}
\begin{Highlighting}[]
\NormalTok{notas=}\KeywordTok{c}\NormalTok{(G1,G2)}
\NormalTok{grupo=}\KeywordTok{as.factor}\NormalTok{(}\KeywordTok{c}\NormalTok{(}\KeywordTok{rep}\NormalTok{(}\DecValTok{1}\NormalTok{,}\KeywordTok{length}\NormalTok{(G1)),}\KeywordTok{rep}\NormalTok{(}\DecValTok{2}\NormalTok{,}\KeywordTok{length}\NormalTok{(G1))))}
\KeywordTok{leveneTest}\NormalTok{(notas}\OperatorTok{~}\NormalTok{grupo)}
\end{Highlighting}
\end{Shaded}

\begin{verbatim}
## Levene's Test for Homogeneity of Variance (center = median)
##       Df F value Pr(>F)
## group  1  1.8029 0.1825
##       98
\end{verbatim}

\begin{Shaded}
\begin{Highlighting}[]
\KeywordTok{t.test}\NormalTok{(G1,G2,}\DataTypeTok{var.equal =} \OtherTok{TRUE}\NormalTok{)}
\end{Highlighting}
\end{Shaded}

\begin{verbatim}
## 
##  Two Sample t-test
## 
## data:  G1 and G2
## t = -6.2159, df = 98, p-value = 0.00000001248
## alternative hypothesis: true difference in means is not equal to 0
## 95 percent confidence interval:
##  -2.251691 -1.161889
## sample estimates:
## mean of x mean of y 
##  9.759293 11.466083
\end{verbatim}

\begin{Shaded}
\begin{Highlighting}[]
\KeywordTok{t.test}\NormalTok{(G1,G2,}\DataTypeTok{var.equal =} \OtherTok{FALSE}\NormalTok{)}
\end{Highlighting}
\end{Shaded}

\begin{verbatim}
## 
##  Welch Two Sample t-test
## 
## data:  G1 and G2
## t = -6.2159, df = 89.996, p-value = 0.00000001562
## alternative hypothesis: true difference in means is not equal to 0
## 95 percent confidence interval:
##  -2.252298 -1.161282
## sample estimates:
## mean of x mean of y 
##  9.759293 11.466083
\end{verbatim}

\hypertarget{problema-2-contraste-dos-muestras}{%
\subsection{Problema 2 : Contraste dos
muestras}\label{problema-2-contraste-dos-muestras}}

Simulamos dos muestras con las funciones siguientes

\begin{Shaded}
\begin{Highlighting}[]
\KeywordTok{set.seed}\NormalTok{(}\DecValTok{2020}\NormalTok{)}
\NormalTok{x1=}\KeywordTok{rnorm}\NormalTok{(}\DecValTok{100}\NormalTok{,}\DataTypeTok{mean =} \DecValTok{10}\NormalTok{,}\DataTypeTok{sd=}\DecValTok{2}\NormalTok{)}
\NormalTok{x2=}\KeywordTok{rnorm}\NormalTok{(}\DecValTok{100}\NormalTok{,}\DataTypeTok{mean =} \DecValTok{8}\NormalTok{,}\DataTypeTok{sd=}\DecValTok{4}\NormalTok{)}
\end{Highlighting}
\end{Shaded}

Dibujamos estos gráficos

\begin{Shaded}
\begin{Highlighting}[]
\KeywordTok{boxplot}\NormalTok{(x1,x2)}
\end{Highlighting}
\end{Shaded}

\includegraphics{Entrega3_ENUNCIADO_SOLUCION_files/figure-latex/unnamed-chunk-4-1.pdf}

\begin{Shaded}
\begin{Highlighting}[]
\KeywordTok{library}\NormalTok{(car)}
\KeywordTok{par}\NormalTok{(}\DataTypeTok{mfrow=}\KeywordTok{c}\NormalTok{(}\DecValTok{1}\NormalTok{,}\DecValTok{2}\NormalTok{))}
\KeywordTok{qqPlot}\NormalTok{(x1)}
\end{Highlighting}
\end{Shaded}

\begin{verbatim}
## [1] 18 64
\end{verbatim}

\begin{Shaded}
\begin{Highlighting}[]
\KeywordTok{qqPlot}\NormalTok{(x2)}
\end{Highlighting}
\end{Shaded}

\includegraphics{Entrega3_ENUNCIADO_SOLUCION_files/figure-latex/unnamed-chunk-4-2.pdf}

\begin{verbatim}
## [1] 50 39
\end{verbatim}

\begin{Shaded}
\begin{Highlighting}[]
\KeywordTok{par}\NormalTok{(}\DataTypeTok{mfrow=}\KeywordTok{c}\NormalTok{(}\DecValTok{1}\NormalTok{,}\DecValTok{1}\NormalTok{))}
\end{Highlighting}
\end{Shaded}

Realizamos algunos contrastes de hipótesis de igual de medias entre
ambas muestras

\begin{Shaded}
\begin{Highlighting}[]
\KeywordTok{t.test}\NormalTok{(x1,x2,}\DataTypeTok{var.equal =} \OtherTok{TRUE}\NormalTok{,}\DataTypeTok{alternative =} \StringTok{"greater"}\NormalTok{)}
\end{Highlighting}
\end{Shaded}

\begin{verbatim}
## 
##  Two Sample t-test
## 
## data:  x1 and x2
## t = 5.3009, df = 198, p-value = 0.0000001531
## alternative hypothesis: true difference in means is greater than 0
## 95 percent confidence interval:
##  1.844757      Inf
## sample estimates:
## mean of x mean of y 
## 10.217784  7.537402
\end{verbatim}

\begin{Shaded}
\begin{Highlighting}[]
\KeywordTok{t.test}\NormalTok{(x1,x2,}\DataTypeTok{var.equal =} \OtherTok{FALSE}\NormalTok{,}\DataTypeTok{alternative =} \StringTok{"two.sided"}\NormalTok{)}
\end{Highlighting}
\end{Shaded}

\begin{verbatim}
## 
##  Welch Two Sample t-test
## 
## data:  x1 and x2
## t = 5.3009, df = 144.56, p-value = 0.0000004221
## alternative hypothesis: true difference in means is not equal to 0
## 95 percent confidence interval:
##  1.680966 3.679797
## sample estimates:
## mean of x mean of y 
## 10.217784  7.537402
\end{verbatim}

\begin{Shaded}
\begin{Highlighting}[]
\KeywordTok{t.test}\NormalTok{(x1,x2,}\DataTypeTok{var.equal =} \OtherTok{TRUE}\NormalTok{)}
\end{Highlighting}
\end{Shaded}

\begin{verbatim}
## 
##  Two Sample t-test
## 
## data:  x1 and x2
## t = 5.3009, df = 198, p-value = 0.0000003061
## alternative hypothesis: true difference in means is not equal to 0
## 95 percent confidence interval:
##  1.683238 3.677526
## sample estimates:
## mean of x mean of y 
## 10.217784  7.537402
\end{verbatim}

Se pide

\begin{enumerate}
\def\labelenumi{\arabic{enumi}.}
\tightlist
\item
  ¿Cuál es la distribución y los parámetros de las muestras generadas?
\item
  ¿Qué muestran y cuál es la interpretación de los gráficos?
\item
  ¿Qué test contrasta si hay evidencia a favor de que las medias
  poblacionales de las notas en cada grupo sean distintas? Di qué código
  de los anteriores resuelve este test.
\item
  Para el test del apartado anterior dad las hipótesis nula y
  alternativa y redactar la conclusión del contraste.
\end{enumerate}

\hypertarget{problema-3-bondad-de-ajuste.-la-ley-de-benford}{%
\subsection{Problema 3 : Bondad de ajuste. La ley de
Benford}\label{problema-3-bondad-de-ajuste.-la-ley-de-benford}}

La ley de Benford es una distribución discreta que siguen las
frecuencias de los primero dígitos significativos (de 1 a 9) de algunas
series de datos curiosas.

Sea una v.a. X con dominio \(D_X=\left\{1,2,3,4,5,6,7,8,9\right\}\)
diremos que sigue una ley de Benford si

\[P(X=x)=\log_{10} \left(1+\frac{1}{x}\right)\mbox{ para } x\in \left\{1,2,3,4,5,6,7,8,9\right\}.\]

Concretamente lo podemos hacer así

\begin{Shaded}
\begin{Highlighting}[]
\NormalTok{prob=}\KeywordTok{log10}\NormalTok{(}\DecValTok{1}\OperatorTok{+}\DecValTok{1}\OperatorTok{/}\KeywordTok{c}\NormalTok{(}\DecValTok{1}\OperatorTok{:}\DecValTok{9}\NormalTok{))}
\NormalTok{prob}
\end{Highlighting}
\end{Shaded}

\begin{verbatim}
## [1] 0.30103000 0.17609126 0.12493874 0.09691001 0.07918125 0.06694679 0.05799195
## [8] 0.05115252 0.04575749
\end{verbatim}

\begin{Shaded}
\begin{Highlighting}[]
\NormalTok{MM=}\KeywordTok{rbind}\NormalTok{(}\KeywordTok{c}\NormalTok{(}\DecValTok{1}\OperatorTok{:}\DecValTok{9}\NormalTok{),prob)}
\NormalTok{df=}\KeywordTok{data.frame}\NormalTok{(}\KeywordTok{rbind}\NormalTok{(prob))}
\CommentTok{# Y hacemos una bonita tabla}
\KeywordTok{colnames}\NormalTok{(df)=}\KeywordTok{paste}\NormalTok{(}\StringTok{"Díg."}\NormalTok{,}\KeywordTok{c}\NormalTok{(}\DecValTok{1}\OperatorTok{:}\DecValTok{9}\NormalTok{),}\DataTypeTok{sep =}\StringTok{" "}\NormalTok{)}
\NormalTok{knitr}\OperatorTok{::}\KeywordTok{kable}\NormalTok{(df,}\DataTypeTok{format =}\StringTok{'markdown'}\NormalTok{)}
\end{Highlighting}
\end{Shaded}

\begin{longtable}[]{@{}lrrrrrrrrr@{}}
\toprule
\begin{minipage}[b]{0.04\columnwidth}\raggedright
\strut
\end{minipage} & \begin{minipage}[b]{0.06\columnwidth}\raggedleft
Díg. 1\strut
\end{minipage} & \begin{minipage}[b]{0.08\columnwidth}\raggedleft
Díg. 2\strut
\end{minipage} & \begin{minipage}[b]{0.08\columnwidth}\raggedleft
Díg. 3\strut
\end{minipage} & \begin{minipage}[b]{0.06\columnwidth}\raggedleft
Díg. 4\strut
\end{minipage} & \begin{minipage}[b]{0.08\columnwidth}\raggedleft
Díg. 5\strut
\end{minipage} & \begin{minipage}[b]{0.08\columnwidth}\raggedleft
Díg. 6\strut
\end{minipage} & \begin{minipage}[b]{0.08\columnwidth}\raggedleft
Díg. 7\strut
\end{minipage} & \begin{minipage}[b]{0.08\columnwidth}\raggedleft
Díg. 8\strut
\end{minipage} & \begin{minipage}[b]{0.08\columnwidth}\raggedleft
Díg. 9\strut
\end{minipage}\tabularnewline
\midrule
\endhead
\begin{minipage}[t]{0.04\columnwidth}\raggedright
prob\strut
\end{minipage} & \begin{minipage}[t]{0.06\columnwidth}\raggedleft
0.30103\strut
\end{minipage} & \begin{minipage}[t]{0.08\columnwidth}\raggedleft
0.1760913\strut
\end{minipage} & \begin{minipage}[t]{0.08\columnwidth}\raggedleft
0.1249387\strut
\end{minipage} & \begin{minipage}[t]{0.06\columnwidth}\raggedleft
0.09691\strut
\end{minipage} & \begin{minipage}[t]{0.08\columnwidth}\raggedleft
0.0791812\strut
\end{minipage} & \begin{minipage}[t]{0.08\columnwidth}\raggedleft
0.0669468\strut
\end{minipage} & \begin{minipage}[t]{0.08\columnwidth}\raggedleft
0.0579919\strut
\end{minipage} & \begin{minipage}[t]{0.08\columnwidth}\raggedleft
0.0511525\strut
\end{minipage} & \begin{minipage}[t]{0.08\columnwidth}\raggedleft
0.0457575\strut
\end{minipage}\tabularnewline
\bottomrule
\end{longtable}

En general esta distribución se suele encontrar en tablas de datos de
resultados de observaciones de funciones científicas, contabilidades,
cocientes de algunas distribuciones \ldots{}

Por ejemplo se dice que las potencias de números enteros siguen esa
distribución. Probemos con las potencias de 2. El siguiente código
calcula las potencias de 2 de 1 a 1000 y extrae los tres primeros
dígitos.

\begin{Shaded}
\begin{Highlighting}[]
\CommentTok{# R pasa los enteros  muy grande a reales. Para nuestros propósitos }
\CommentTok{# es suficiente para extraer los tres primeros dígitos.}
\NormalTok{muestra_pot_}\DecValTok{2}\NormalTok{_3digitos=}\KeywordTok{str_sub}\NormalTok{(}\KeywordTok{as.character}\NormalTok{(}\DecValTok{2}\OperatorTok{^}\KeywordTok{c}\NormalTok{(}\DecValTok{1}\OperatorTok{:}\DecValTok{1000}\NormalTok{)),}\DecValTok{1}\NormalTok{,}\DecValTok{3}\NormalTok{)}
\KeywordTok{head}\NormalTok{(muestra_pot_}\DecValTok{2}\NormalTok{_3digitos)}
\end{Highlighting}
\end{Shaded}

\begin{verbatim}
## [1] "2"  "4"  "8"  "16" "32" "64"
\end{verbatim}

\begin{Shaded}
\begin{Highlighting}[]
\KeywordTok{tail}\NormalTok{(muestra_pot_}\DecValTok{2}\NormalTok{_3digitos)}
\end{Highlighting}
\end{Shaded}

\begin{verbatim}
## [1] "334" "669" "133" "267" "535" "107"
\end{verbatim}

\begin{Shaded}
\begin{Highlighting}[]
\CommentTok{#Construimos un data frame con tres columnas que nos dan el primer, }
\CommentTok{#segundo y tercer dígito respectivamente.}
\NormalTok{df_digitos=}\KeywordTok{data.frame}\NormalTok{(muestra_pot_}\DecValTok{2}\NormalTok{_3digitos,}
                      \DataTypeTok{primer_digito=}\KeywordTok{as.integer}\NormalTok{(}
                        \KeywordTok{substring}\NormalTok{(muestra_pot_}\DecValTok{2}\NormalTok{_3digitos, }\DecValTok{1}\NormalTok{, }\DecValTok{1}\NormalTok{)),}
                      \DataTypeTok{segundo_digito=}\KeywordTok{as.integer}\NormalTok{(}
                        \KeywordTok{substring}\NormalTok{(muestra_pot_}\DecValTok{2}\NormalTok{_3digitos, }\DecValTok{2}\NormalTok{, }\DecValTok{2}\NormalTok{)),}
                      \DataTypeTok{tercer_digito=}\KeywordTok{as.integer}\NormalTok{(}
                        \KeywordTok{substring}\NormalTok{(muestra_pot_}\DecValTok{2}\NormalTok{_3digitos, }\DecValTok{3}\NormalTok{, }\DecValTok{3}\NormalTok{)))}
\KeywordTok{head}\NormalTok{(df_digitos)}
\end{Highlighting}
\end{Shaded}

\begin{verbatim}
##   muestra_pot_2_3digitos primer_digito segundo_digito tercer_digito
## 1                      2             2             NA            NA
## 2                      4             4             NA            NA
## 3                      8             8             NA            NA
## 4                     16             1              6            NA
## 5                     32             3              2            NA
## 6                     64             6              4            NA
\end{verbatim}

Notad que los NA en el segundo y el tercer dígito corresponden a número
con uno o dos dígitos.

Se pide:

\begin{enumerate}
\def\labelenumi{\arabic{enumi}.}
\tightlist
\item
  Contrastad con un test \(\chi^2\) que el primer dígito sigue una ley
  de Benford. Notad que el primer dígito no puede ser 0. Resolved
  manualmente y con una función de \texttt{R}.\\
\item
  Contrastad con un test \(\chi^2\) que el segundo dígito sigue una ley
  de uniforme discreta. Notad que ahora si puede ser 0. Resolved con
  funciones de \texttt{R}.\\
\item
  Contrastad con un test \(\chi^2\) que el tercer dígito sigue una ley
  de uniforme discreta. Notad que ahora si puede ser 0. Resolved con
  manualmente calculado las frecuencias esperadas y observadas, el
  estadístico de contraste y el \(p\)-valor utilizando \texttt{R}.
  Comprobad que vuestros resultados coinciden con los de la función de
  \texttt{R} que calcula este contraste.\\
\item
  Dibujad con \texttt{R} para los apartados 1 y 2 los diagramas de
  frecuencias esperados y observados. Comentad estos gráficos
\end{enumerate}

\hypertarget{pre-soluciuxf3n}{%
\subsubsection{Pre Solución}\label{pre-soluciuxf3n}}

\begin{Shaded}
\begin{Highlighting}[]
\NormalTok{prob=}\KeywordTok{log10}\NormalTok{(}\DecValTok{1}\OperatorTok{+}\DecValTok{1}\OperatorTok{/}\NormalTok{(}\DecValTok{1}\OperatorTok{:}\DecValTok{9}\NormalTok{))}
\NormalTok{prob_benford=prob}
\NormalTok{n=}\DecValTok{1000}
\NormalTok{frec_esp_benford=n}\OperatorTok{*}\NormalTok{prob_benford}
\NormalTok{frec_esp_benford}
\end{Highlighting}
\end{Shaded}

\begin{verbatim}
## [1] 301.03000 176.09126 124.93874  96.91001  79.18125  66.94679  57.99195
## [8]  51.15252  45.75749
\end{verbatim}

\begin{Shaded}
\begin{Highlighting}[]
\NormalTok{frec_obs_primer=}\KeywordTok{table}\NormalTok{(df_digitos}\OperatorTok{$}\NormalTok{primer_digito)}
\NormalTok{frec_obs_primer}
\end{Highlighting}
\end{Shaded}

\begin{verbatim}
## 
##   1   2   3   4   5   6   7   8   9 
## 301 176 125  97  79  69  56  52  45
\end{verbatim}

\begin{Shaded}
\begin{Highlighting}[]
\NormalTok{chi2_est=}\KeywordTok{sum}\NormalTok{((frec_obs_primer}\OperatorTok{-}\NormalTok{frec_esp_benford)}\OperatorTok{^}\DecValTok{2}\OperatorTok{/}\NormalTok{frec_esp_benford)}
\NormalTok{chi2_est}
\end{Highlighting}
\end{Shaded}

\begin{verbatim}
## [1] 0.1585506
\end{verbatim}

\begin{Shaded}
\begin{Highlighting}[]
\KeywordTok{chisq.test}\NormalTok{(frec_obs_primer,}\DataTypeTok{p=}\NormalTok{prob_benford)}
\end{Highlighting}
\end{Shaded}

\begin{verbatim}
## 
##  Chi-squared test for given probabilities
## 
## data:  frec_obs_primer
## X-squared = 0.15855, df = 8, p-value = 1
\end{verbatim}

Una gráfica comparando las frecuencias

\begin{Shaded}
\begin{Highlighting}[]
\KeywordTok{barplot}\NormalTok{(}\KeywordTok{rbind}\NormalTok{(frec_esp_benford,frec_obs_primer),}\DataTypeTok{beside=}\OtherTok{TRUE}\NormalTok{,}\DataTypeTok{col=}\KeywordTok{c}\NormalTok{(}\StringTok{"red"}\NormalTok{,}\StringTok{"blue"}\NormalTok{))}
\end{Highlighting}
\end{Shaded}

\includegraphics{Entrega3_ENUNCIADO_SOLUCION_files/figure-latex/unnamed-chunk-6-1.pdf}

\begin{Shaded}
\begin{Highlighting}[]
\CommentTok{# añadir legend}
\end{Highlighting}
\end{Shaded}

\begin{Shaded}
\begin{Highlighting}[]
\NormalTok{prob_unif=}\KeywordTok{rep}\NormalTok{(}\DecValTok{1}\OperatorTok{/}\DecValTok{10}\NormalTok{,}\DecValTok{10}\NormalTok{)}
\NormalTok{prob_unif}
\end{Highlighting}
\end{Shaded}

\begin{verbatim}
##  [1] 0.1 0.1 0.1 0.1 0.1 0.1 0.1 0.1 0.1 0.1
\end{verbatim}

\begin{Shaded}
\begin{Highlighting}[]
\NormalTok{segundo_digito=}\KeywordTok{na.omit}\NormalTok{(df_digitos}\OperatorTok{$}\NormalTok{segundo_digito)}
\NormalTok{n=}\KeywordTok{length}\NormalTok{(segundo_digito) }
\NormalTok{n}
\end{Highlighting}
\end{Shaded}

\begin{verbatim}
## [1] 997
\end{verbatim}

\begin{Shaded}
\begin{Highlighting}[]
\NormalTok{frec_esp_uniforme=n}\OperatorTok{*}\NormalTok{prob_unif}
\NormalTok{frec_esp_uniforme }
\end{Highlighting}
\end{Shaded}

\begin{verbatim}
##  [1] 99.7 99.7 99.7 99.7 99.7 99.7 99.7 99.7 99.7 99.7
\end{verbatim}

\begin{Shaded}
\begin{Highlighting}[]
\NormalTok{frec_obs_segundo=}\KeywordTok{table}\NormalTok{(segundo_digito)}
\NormalTok{frec_obs_segundo}
\end{Highlighting}
\end{Shaded}

\begin{verbatim}
## segundo_digito
##   0   1   2   3   4   5   6   7   8   9 
## 121 112 109 108  98  95  94  91  83  86
\end{verbatim}

\begin{Shaded}
\begin{Highlighting}[]
\NormalTok{chi2_est=}\KeywordTok{sum}\NormalTok{((frec_obs_segundo}\OperatorTok{-}\NormalTok{frec_esp_uniforme)}\OperatorTok{^}\DecValTok{2}\OperatorTok{/}\NormalTok{frec_esp_uniforme)}
\NormalTok{chi2_est}
\end{Highlighting}
\end{Shaded}

\begin{verbatim}
## [1] 13.64193
\end{verbatim}

\begin{Shaded}
\begin{Highlighting}[]
\DecValTok{1}\OperatorTok{-}\KeywordTok{pchisq}\NormalTok{(chi2_est,}\DecValTok{10-1}\NormalTok{)}
\end{Highlighting}
\end{Shaded}

\begin{verbatim}
## [1] 0.1356449
\end{verbatim}

\begin{Shaded}
\begin{Highlighting}[]
\KeywordTok{pchisq}\NormalTok{(chi2_est,}\DecValTok{10-1}\NormalTok{,}\DataTypeTok{lower.tail =} \OtherTok{FALSE}\NormalTok{)}
\end{Highlighting}
\end{Shaded}

\begin{verbatim}
## [1] 0.1356449
\end{verbatim}

\begin{Shaded}
\begin{Highlighting}[]
\KeywordTok{chisq.test}\NormalTok{(frec_obs_segundo,}\DataTypeTok{p=}\NormalTok{prob_unif)}
\end{Highlighting}
\end{Shaded}

\begin{verbatim}
## 
##  Chi-squared test for given probabilities
## 
## data:  frec_obs_segundo
## X-squared = 13.642, df = 9, p-value = 0.1356
\end{verbatim}

\begin{Shaded}
\begin{Highlighting}[]
\KeywordTok{barplot}\NormalTok{(}\KeywordTok{rbind}\NormalTok{(frec_esp_uniforme,frec_obs_segundo),}\DataTypeTok{beside=}\OtherTok{TRUE}\NormalTok{,}\DataTypeTok{col=}\KeywordTok{c}\NormalTok{(}\StringTok{"red"}\NormalTok{,}\StringTok{"blue"}\NormalTok{))}
\end{Highlighting}
\end{Shaded}

\includegraphics{Entrega3_ENUNCIADO_SOLUCION_files/figure-latex/unnamed-chunk-8-1.pdf}

\begin{Shaded}
\begin{Highlighting}[]
\CommentTok{# añadir legend}
\end{Highlighting}
\end{Shaded}

\hypertarget{problema-4-homegeneidad-e-independencia}{%
\subsection{Problema 4 : Homegeneidad e
independencia}\label{problema-4-homegeneidad-e-independencia}}

Queremos analiza los resultados de aprendizaje con tres tecnologías.
Para ello se seleccionan 3 muestras de 50 estudiantes y se les somete a
evaluación después de un curso.

\begin{Shaded}
\begin{Highlighting}[]
\KeywordTok{set.seed}\NormalTok{(}\DecValTok{2020}\NormalTok{)}
\NormalTok{nota=}\KeywordTok{factor}\NormalTok{(}\KeywordTok{sample}\NormalTok{(}\KeywordTok{c}\NormalTok{(}\DecValTok{1}\NormalTok{,}\DecValTok{2}\NormalTok{,}\DecValTok{3}\NormalTok{,}\DecValTok{4}\NormalTok{),}\DataTypeTok{p=}\KeywordTok{c}\NormalTok{(}\FloatTok{0.1}\NormalTok{,}\FloatTok{0.4}\NormalTok{,}\FloatTok{0.3}\NormalTok{,}\FloatTok{0.2}\NormalTok{),}\DataTypeTok{replace=}\OtherTok{TRUE}\NormalTok{,}\DataTypeTok{size=}\DecValTok{150}\NormalTok{),}
            \DataTypeTok{labels=}\KeywordTok{c}\NormalTok{(}\StringTok{"S"}\NormalTok{,}\StringTok{"A"}\NormalTok{,}\StringTok{"N"}\NormalTok{,}\StringTok{"E"}\NormalTok{))}
\NormalTok{tecnologia=}\KeywordTok{rep}\NormalTok{(}\KeywordTok{c}\NormalTok{(}\StringTok{"Mathematica"}\NormalTok{,}\StringTok{"R"}\NormalTok{,}\StringTok{"Python"}\NormalTok{),}\DataTypeTok{each=}\DecValTok{50}\NormalTok{)}
\NormalTok{frec=}\KeywordTok{table}\NormalTok{(nota,tecnologia)}
\NormalTok{frec}
\end{Highlighting}
\end{Shaded}

\begin{verbatim}
##     tecnologia
## nota Mathematica Python  R
##    S           7      6  2
##    A          18     15 22
##    N          15     20 18
##    E          10      9  8
\end{verbatim}

\begin{Shaded}
\begin{Highlighting}[]
\NormalTok{col_frec=}\KeywordTok{colSums}\NormalTok{(frec)}
\NormalTok{col_frec}
\end{Highlighting}
\end{Shaded}

\begin{verbatim}
## Mathematica      Python           R 
##          50          50          50
\end{verbatim}

\begin{Shaded}
\begin{Highlighting}[]
\NormalTok{row_frec=}\KeywordTok{rowSums}\NormalTok{(frec)}
\NormalTok{row_frec}
\end{Highlighting}
\end{Shaded}

\begin{verbatim}
##  S  A  N  E 
## 15 55 53 27
\end{verbatim}

\begin{Shaded}
\begin{Highlighting}[]
\NormalTok{N=}\KeywordTok{sum}\NormalTok{(frec)}
\NormalTok{teoricas=row_frec}\OperatorTok\KeywordTok{t}\NormalTok{(col_frec)}\OperatorTok{/}\NormalTok{N}
\NormalTok{teoricas}
\end{Highlighting}
\end{Shaded}

\begin{verbatim}
##      Mathematica   Python        R
## [1,]     5.00000  5.00000  5.00000
## [2,]    18.33333 18.33333 18.33333
## [3,]    17.66667 17.66667 17.66667
## [4,]     9.00000  9.00000  9.00000
\end{verbatim}

\begin{Shaded}
\begin{Highlighting}[]
\KeywordTok{dim}\NormalTok{(frec)}
\end{Highlighting}
\end{Shaded}

\begin{verbatim}
## [1] 4 3
\end{verbatim}

\begin{Shaded}
\begin{Highlighting}[]
\KeywordTok{dim}\NormalTok{(teoricas)}
\end{Highlighting}
\end{Shaded}

\begin{verbatim}
## [1] 4 3
\end{verbatim}

\begin{Shaded}
\begin{Highlighting}[]
\KeywordTok{sum}\NormalTok{((frec}\OperatorTok{-}\NormalTok{teoricas)}\OperatorTok{^}\DecValTok{2}\OperatorTok{/}\NormalTok{teoricas)}
\end{Highlighting}
\end{Shaded}

\begin{verbatim}
## [1] 5.084658
\end{verbatim}

\begin{Shaded}
\begin{Highlighting}[]
\KeywordTok{chisq.test}\NormalTok{(}\KeywordTok{table}\NormalTok{(nota,tecnologia))}
\end{Highlighting}
\end{Shaded}

\begin{verbatim}
## 
##  Pearson's Chi-squared test
## 
## data:  table(nota, tecnologia)
## X-squared = 5.0847, df = 6, p-value = 0.533
\end{verbatim}

Se pide

\begin{enumerate}
\def\labelenumi{\arabic{enumi}.}
\tightlist
\item
  Discutid si hacemos un contraste de independencia o de homogeneidad de
  las distribuciones de las notas por tecnología. Escribid las hipótesis
  del contraste.
\item
  Interpretad la función \texttt{chisq.test} y resolved el contraste.
\item
  Interpretad \texttt{teoricas=row\_frec\%*\%t(col\_frec)/N} reproducid
  manualmente el segundo resultado de la primera fila.
\end{enumerate}

\hypertarget{problema-5-anova-notas-numuxe9ricas-de-tres-grupos.}{%
\subsection{Problema 5 : ANOVA notas numéricas de tres
grupos.}\label{problema-5-anova-notas-numuxe9ricas-de-tres-grupos.}}

El siguiente código nos da las notas numéricas (variable \texttt{nota})
de los mismos ejercicios para tres tecnologías en tres muestra
independientes de estudiantes de estas tres tecnologías diferentes

\begin{Shaded}
\begin{Highlighting}[]
\KeywordTok{head}\NormalTok{(nota)}
\end{Highlighting}
\end{Shaded}

\begin{verbatim}
## [1] 79.424303 77.538709 42.549421 41.739852  0.086642 88.014337
\end{verbatim}

\begin{Shaded}
\begin{Highlighting}[]
\KeywordTok{library}\NormalTok{(nortest)}
\KeywordTok{lillie.test}\NormalTok{(nota[tecnologia}\OperatorTok{==}\StringTok{"Mathematica"}\NormalTok{])}
\end{Highlighting}
\end{Shaded}

\begin{verbatim}
## 
##  Lilliefors (Kolmogorov-Smirnov) normality test
## 
## data:  nota[tecnologia == "Mathematica"]
## D = 0.08739, p-value = 0.4436
\end{verbatim}

\begin{Shaded}
\begin{Highlighting}[]
\KeywordTok{lillie.test}\NormalTok{(nota[tecnologia}\OperatorTok{==}\StringTok{"R"}\NormalTok{])}
\end{Highlighting}
\end{Shaded}

\begin{verbatim}
## 
##  Lilliefors (Kolmogorov-Smirnov) normality test
## 
## data:  nota[tecnologia == "R"]
## D = 0.082139, p-value = 0.5449
\end{verbatim}

\begin{Shaded}
\begin{Highlighting}[]
\KeywordTok{lillie.test}\NormalTok{(nota[tecnologia}\OperatorTok{==}\StringTok{"Python"}\NormalTok{])}
\end{Highlighting}
\end{Shaded}

\begin{verbatim}
## 
##  Lilliefors (Kolmogorov-Smirnov) normality test
## 
## data:  nota[tecnologia == "Python"]
## D = 0.089681, p-value = 0.4019
\end{verbatim}

\begin{Shaded}
\begin{Highlighting}[]
\KeywordTok{lillie.test}\NormalTok{(nota)}
\end{Highlighting}
\end{Shaded}

\begin{verbatim}
## 
##  Lilliefors (Kolmogorov-Smirnov) normality test
## 
## data:  nota
## D = 0.056381, p-value = 0.2885
\end{verbatim}

\begin{Shaded}
\begin{Highlighting}[]
\KeywordTok{bartlett.test}\NormalTok{(nota}\OperatorTok{~}\NormalTok{tecnologia)}
\end{Highlighting}
\end{Shaded}

\begin{verbatim}
## 
##  Bartlett test of homogeneity of variances
## 
## data:  nota by tecnologia
## Bartlett's K-squared = 0.50309, df = 2, p-value = 0.7776
\end{verbatim}

\begin{Shaded}
\begin{Highlighting}[]
\KeywordTok{library}\NormalTok{(car)}
\KeywordTok{leveneTest}\NormalTok{(nota}\OperatorTok{~}\KeywordTok{as.factor}\NormalTok{(tecnologia))}
\end{Highlighting}
\end{Shaded}

\begin{verbatim}
## Levene's Test for Homogeneity of Variance (center = median)
##        Df F value Pr(>F)
## group   2  0.3881  0.679
##       147
\end{verbatim}

\begin{Shaded}
\begin{Highlighting}[]
\NormalTok{sol_aov=}\KeywordTok{aov}\NormalTok{(nota}\OperatorTok{~}\KeywordTok{as.factor}\NormalTok{(tecnologia))}
\NormalTok{sol_aov}
\end{Highlighting}
\end{Shaded}

\begin{verbatim}
## Call:
##    aov(formula = nota ~ as.factor(tecnologia))
## 
## Terms:
##                 as.factor(tecnologia) Residuals
## Sum of Squares                 837.39 123445.06
## Deg. of Freedom                     2       147
## 
## Residual standard error: 28.97865
## Estimated effects may be unbalanced
\end{verbatim}

Del \texttt{summary(sol\_aov)} os damos la salida a falta de algunos de
los valores

\begin{verbatim}
> summary(sol_aov)
                      Df Sum Sq Mean Sq F value Pr(>F)
as.factor(tecnologia) ---    837   418.7   ---  ---
Residuals             --- 123445   839.8                          
\end{verbatim}

\begin{Shaded}
\begin{Highlighting}[]
\KeywordTok{pairwise.t.test}\NormalTok{(nota,}\KeywordTok{as.factor}\NormalTok{(tecnologia),}\DataTypeTok{p.adjust.method =} \StringTok{"none"}\NormalTok{)}
\end{Highlighting}
\end{Shaded}

\begin{verbatim}
## 
##  Pairwise comparisons using t tests with pooled SD 
## 
## data:  nota and as.factor(tecnologia) 
## 
##        Mathematica Python
## Python 0.35        -     
## R      0.89        0.43  
## 
## P value adjustment method: none
\end{verbatim}

\begin{Shaded}
\begin{Highlighting}[]
\KeywordTok{pairwise.t.test}\NormalTok{(nota,}\KeywordTok{as.factor}\NormalTok{(tecnologia),}\DataTypeTok{p.adjust.method =} \StringTok{"bonferroni"}\NormalTok{)}
\end{Highlighting}
\end{Shaded}

\begin{verbatim}
## 
##  Pairwise comparisons using t tests with pooled SD 
## 
## data:  nota and as.factor(tecnologia) 
## 
##        Mathematica Python
## Python 1           -     
## R      1           1     
## 
## P value adjustment method: bonferroni
\end{verbatim}

Se pide

\begin{enumerate}
\def\labelenumi{\arabic{enumi}.}
\tightlist
\item
  ¿Podemos asegurar que la muestras son normales en cada grupo? ¿y son
  homocedásticas? Sea cual sea la respuesta justificad que parte del
  código la confirma.
\item
  La función \texttt{aov} que test calcula. Escribid formalmente la
  hipótesis nula y la alternativa.
\item
  Calcula la tabla de ANOVA y resuelve el test.
\item
  ¿Qué contrates realiza la función \texttt{pairwise.t.test}? Utilizando
  los resultados anteriores aplicad e interpretad los contrates del
  apartado anterior utilizando el ajuste de Holm.
\end{enumerate}

\hypertarget{problema-6-anova-comparaciuxf3n-de-las-tasas-de-interuxe9s-para-la-compra-de-coches-entre-seis-ciudades.}{%
\subsection{Problema 6 : ANOVA Comparación de las tasas de interés para
la compra de coches entre seis
ciudades.}\label{problema-6-anova-comparaciuxf3n-de-las-tasas-de-interuxe9s-para-la-compra-de-coches-entre-seis-ciudades.}}

Consideremos el \texttt{data\ set} \texttt{newcar} accesible desde
\url{https://www.itl.nist.gov/div898/education/anova/newcar.dat} de
\emph{Hoaglin, D., Mosteller, F., and Tukey, J. (1991). Fundamentals of
Exploratory Analysis of Variance. Wiley, New York, page 71.}

Este data set contiene dos columnas:

\begin{itemize}
\tightlist
\item
  Rate (interés): tasa de interés en la compra de coches a crédito
\item
  City (ciudad) : la ciudad en la que se observó la tasa de interés para
  distintos concesionarios (codificada a enteros). Tenemos observaciones
  de 6 ciudades.
\end{itemize}

\begin{Shaded}
\begin{Highlighting}[]
\NormalTok{datos_interes=}\KeywordTok{read.table}\NormalTok{(}
  \StringTok{"https://www.itl.nist.gov/div898/education/anova/newcar.dat"}\NormalTok{,}
  \DataTypeTok{skip=}\DecValTok{25}\NormalTok{)}
\CommentTok{# salto las 25 primeras líneas del fichero,son un preámbulo qiue explica los datos.}
\KeywordTok{names}\NormalTok{(datos_interes)=}\KeywordTok{c}\NormalTok{(}\StringTok{"interes"}\NormalTok{,}\StringTok{"ciudad"}\NormalTok{)}
\KeywordTok{str}\NormalTok{(datos_interes)}
\end{Highlighting}
\end{Shaded}

\begin{verbatim}
## 'data.frame':    54 obs. of  2 variables:
##  $ interes: num  13.8 13.8 13.5 13.5 13 ...
##  $ ciudad : int  1 1 1 1 1 1 1 1 1 2 ...
\end{verbatim}

\begin{Shaded}
\begin{Highlighting}[]
\KeywordTok{boxplot}\NormalTok{(interes}\OperatorTok{~}\NormalTok{ciudad,}\DataTypeTok{data=}\NormalTok{datos_interes)}
\end{Highlighting}
\end{Shaded}

\includegraphics{Entrega3_ENUNCIADO_SOLUCION_files/figure-latex/unnamed-chunk-9-1.pdf}

Se pide:

\begin{enumerate}
\def\labelenumi{\arabic{enumi}.}
\tightlist
\item
  Comentad el código y el diagrama de caja.
\item
  Se trata de contrastar si hay evidencia de que la tasas medias de
  interés por ciudades son distintas. Definid el ANOVA que contrasta
  esta hipótesis y especificar qué condiciones deben cumplir las
  muestras para poder aplicar el ANOVA.\\
\item
  Comprobad las condiciones del ANOVA con un test KS y un test de Levene
  (con código de \texttt{R}). Justificad las conclusiones.\\
\item
  Realizad el contraste de ANOVA (se cumplan las condiciones o no) y
  redactar adecuadamente la conclusión. Tenéis que hacedlo con funciones
  de \texttt{R}.\\
\item
  Se acepte o no la igualdad de medias realizar las comparaciones dos a
  dos con ajustando los \(p\)-valor tanto por Bonferroni como por Holm
  al nivel de significación \(\alpha=0.1\). Redactad las conclusiones
  que se obtienen de las mismas.
\end{enumerate}

\hypertarget{soluciuxf3n}{%
\subsubsection{Solución}\label{soluciuxf3n}}

\begin{Shaded}
\begin{Highlighting}[]
\KeywordTok{summary}\NormalTok{(}\KeywordTok{aov}\NormalTok{(datos_interes}\OperatorTok{$}\NormalTok{interes}\OperatorTok{~}\KeywordTok{as.factor}\NormalTok{(datos_interes}\OperatorTok{$}\NormalTok{ciudad)))}
\end{Highlighting}
\end{Shaded}

\begin{verbatim}
##                                 Df Sum Sq Mean Sq F value  Pr(>F)   
## as.factor(datos_interes$ciudad)  5  10.95  2.1891   4.829 0.00117 **
## Residuals                       48  21.76  0.4533                   
## ---
## Signif. codes:  0 '***' 0.001 '**' 0.01 '*' 0.05 '.' 0.1 ' ' 1
\end{verbatim}

\begin{Shaded}
\begin{Highlighting}[]
\NormalTok{Fest=}\FloatTok{2.1891}\OperatorTok{/}\FloatTok{0.4533}
\NormalTok{Fest}
\end{Highlighting}
\end{Shaded}

\begin{verbatim}
## [1] 4.829252
\end{verbatim}

\begin{Shaded}
\begin{Highlighting}[]
\DecValTok{1}\OperatorTok{-}\KeywordTok{pf}\NormalTok{(Fest,}\DecValTok{5}\NormalTok{,}\DecValTok{48}\NormalTok{)}
\end{Highlighting}
\end{Shaded}

\begin{verbatim}
## [1] 0.001174782
\end{verbatim}

\begin{Shaded}
\begin{Highlighting}[]
\KeywordTok{pf}\NormalTok{(Fest,}\DecValTok{5}\NormalTok{,}\DecValTok{48}\NormalTok{,}\DataTypeTok{lower.tail =} \OtherTok{FALSE}\NormalTok{)}
\end{Highlighting}
\end{Shaded}

\begin{verbatim}
## [1] 0.001174782
\end{verbatim}

\begin{Shaded}
\begin{Highlighting}[]
\KeywordTok{library}\NormalTok{(nortest)}

\KeywordTok{lillie.test}\NormalTok{(datos_interes}\OperatorTok{$}\NormalTok{interes[datos_interes}\OperatorTok{$}\NormalTok{ciudad}\OperatorTok{==}\DecValTok{1}\NormalTok{])}
\end{Highlighting}
\end{Shaded}

\begin{verbatim}
## 
##  Lilliefors (Kolmogorov-Smirnov) normality test
## 
## data:  datos_interes$interes[datos_interes$ciudad == 1]
## D = 0.22384, p-value = 0.2163
\end{verbatim}

\begin{Shaded}
\begin{Highlighting}[]
\CommentTok{# hacerlo para todos}
\KeywordTok{lillie.test}\NormalTok{(datos_interes}\OperatorTok{$}\NormalTok{interes[datos_interes}\OperatorTok{$}\NormalTok{ciudad}\OperatorTok{==}\DecValTok{2}\NormalTok{])}
\end{Highlighting}
\end{Shaded}

\begin{verbatim}
## 
##  Lilliefors (Kolmogorov-Smirnov) normality test
## 
## data:  datos_interes$interes[datos_interes$ciudad == 2]
## D = 0.22884, p-value = 0.1903
\end{verbatim}

\begin{Shaded}
\begin{Highlighting}[]
\KeywordTok{lillie.test}\NormalTok{(datos_interes}\OperatorTok{$}\NormalTok{interes[datos_interes}\OperatorTok{$}\NormalTok{ciudad}\OperatorTok{==}\DecValTok{6}\NormalTok{])}
\end{Highlighting}
\end{Shaded}

\begin{verbatim}
## 
##  Lilliefors (Kolmogorov-Smirnov) normality test
## 
## data:  datos_interes$interes[datos_interes$ciudad == 6]
## D = 0.3494, p-value = 0.002236
\end{verbatim}

\begin{Shaded}
\begin{Highlighting}[]
\CommentTok{# pasar otros test si se rechaza}
\KeywordTok{ad.test}\NormalTok{(datos_interes}\OperatorTok{$}\NormalTok{interes[datos_interes}\OperatorTok{$}\NormalTok{ciudad}\OperatorTok{==}\DecValTok{6}\NormalTok{])}
\end{Highlighting}
\end{Shaded}

\begin{verbatim}
## 
##  Anderson-Darling normality test
## 
## data:  datos_interes$interes[datos_interes$ciudad == 6]
## A = 1.5897, p-value = 0.0001612
\end{verbatim}

\begin{Shaded}
\begin{Highlighting}[]
\KeywordTok{library}\NormalTok{(car)}
\KeywordTok{print}\NormalTok{(}\KeywordTok{leveneTest}\NormalTok{(datos_interes}\OperatorTok{$}\NormalTok{interes}\OperatorTok{~}\KeywordTok{as.factor}\NormalTok{(datos_interes}\OperatorTok{$}\NormalTok{ciudad)))}
\end{Highlighting}
\end{Shaded}

\begin{verbatim}
## Levene's Test for Homogeneity of Variance (center = median)
##       Df F value Pr(>F)
## group  5  1.2797 0.2882
##       48
\end{verbatim}

\begin{Shaded}
\begin{Highlighting}[]
\KeywordTok{bartlett.test}\NormalTok{(datos_interes}\OperatorTok{$}\NormalTok{interes,}\KeywordTok{as.factor}\NormalTok{(datos_interes}\OperatorTok{$}\NormalTok{ciudad))}
\end{Highlighting}
\end{Shaded}

\begin{verbatim}
## 
##  Bartlett test of homogeneity of variances
## 
## data:  datos_interes$interes and as.factor(datos_interes$ciudad)
## Bartlett's K-squared = 6.2355, df = 5, p-value = 0.284
\end{verbatim}

\[
\left\{
\begin{array}{ll}
H_0: &  \sigma^2_1=\sigma^2_2=\sigma^2_3=\sigma^2_4=\sigma^2_5=\sigma^2_6\\
H_1: & \mbox{ no  todas las varianzas son iguales}.
\end{array}
\right.
\]

\hypertarget{problema-7-cuestiones-cortas}{%
\subsection{Problema 7: Cuestiones
cortas}\label{problema-7-cuestiones-cortas}}

\begin{itemize}
\tightlist
\item
  Cuestión 1: Supongamos que conocemos el \(p\)-valor de un contraste.
  Para que valores de nivel de significación \(\alpha\) RECHAZAMOS la
  hipótesis nula.
\item
  Cuestión 2: Hemos realizado un ANOVA de un factor con 3 niveles, y
  hemos obtenido un \(p\)-valor de 0.001. Suponiendo que las poblaciones
  satisfacen las condiciones para que el ANOVA tenga sentido, ¿podemos
  afirmar con un nivel de significación \(\alpha= 0.05\) que las medias
  de los tres niveles son diferentes dos a dos? Justificad la respuesta.
\item
  Cuestión 3: Lanzamos 300 veces un dado de 6 caras de parchís, queremos
  contrastar que los resultados son equiprobables. ¿Cuáles serian las
  frecuencias esperadas o teóricas del contraste?
\item
  Cuestión 4: En un ANOVA de una vía, queremos contrastar si los 6
  niveles de un factor definen poblaciones con la misma media. Sabemos
  que estas seis poblaciones son normales con la misma varianza
  \(\sigma=2\). Estudiamos a 11 individuos de cada nivel y obtenemos que
  \(SS_{Total}=256.6\) y \(SS_{Tr}=60.3\). ¿Qué vale \(SS_E\). ¿Qué
  valor estimamos que tiene \(\sigma^2\)?
\item
  Cuestión 6: Calculad la correlación entre los vectores de datos
  \(x=(1,3,4,4)\), \(y=(2,4,12,6)\).
\item
  Cuestión 7: De estas cuatro matrices, indicad cuáles pueden ser
  matrices de correlaciones, y explicad por qué.
\end{itemize}

\(A=\left(\begin{array}{cc} 1 & 0.8\\-0.8 & 1\end{array}\right)\),
\(B=\left(\begin{array}{cc} 0.8 & 0.6\\0.6 & 0.8\end{array}\right)\),
\(C=\left(\begin{array}{cc} 1 & 0\\0 & 1\end{array}\right)\),
\(D=\left(\begin{array}{cc} 1 & 1.2\\1.2 & 1\end{array}\right)\).

\end{document}
